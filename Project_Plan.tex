\begin{table}[H]
    {\footnotesize
    \caption{Risk Matrix for Overall Project}
    \centering
    \begin{tabularx}{\linewidth}{XXXX}
    \toprule
    \textbf{Risk} & \textbf{Cause} & \textbf{Impact} & \textbf{Mitigation} \\ \midrule
    Team Miscommunication and Task Delegation & As team management changes year to year, expectations can fluctuate and information can be lost in the transition of power between project teams & 1) This issue creates a void in understanding between team managers and subteam technical members about actual responsibilities. This can lead to confusion, frustration, and loss of interest in the project.  2. Poor task delegation can lead to team members leaving as they feel that they have no work to do or purpose for being on the team. In parallel, task delegation requires a large amount of work on the manager’s part to stay updated with each subteam and cross-team issues. & 1.) Leadership in upper societal management must reflect what is expected of project team managers such as clear safety and ethical decision making. Written documentation about the roles and a “lessons learned” minimizes repeated failures. 2.) Team management hosts quick and productive meetings to ensure continuity and communication across the subteams. 3. Management trusting that team members can make progress on the various subsystems creates a healthy sense of ownership and responsibility for each team member instead of micromanagement. \\
    Part Order Delays & The society’s part ordering system is currently being reorganized due to shifting university policy & 1) Manufacturing and assembling delays that require members to put in long hours and increase fatigue induced mistakes 2. Testing delays that make it difficult to adequately qualify the design & 1.) Find alternative sources of ordering parts to minimize construction and testing downtime 2.) Perform more unit testing so that integrated testing can occur faster when parts have arrived 3.) Plan for these delays in test and integration schedules by doubling or tripling “ideal” schedules \\
    Team Size and Workload imbalance & The team size and available human work power is constantly changing week to week due to tests, vacations, or other situations & 1) Pushes work on reports and other aspects of the project back, reducing quality and quantity of work 2. Project progress is difficult to track 3. Project areas can be neglected due to lack of skill set or team members to satisfy workload & 1.) Discuss schedules and set due dates that account for holidays, exam schedules, and unexpected setbacks 2.) Have smaller, more rapid group meetings where the most pressing design aspects are discussed 3.) Have a consistent reporting schedule for subteam progress 4.) Identify critical systems early on and ensure that there are specific members that are responsible for their delivery \\
    Manufacturing Facility Unavailability & The team will need to work with a machine shop on campus in order to get parts manufactured. During the school year, these facilities can be busy with research orders and jobs for various other organizations on campus that take precedence over a student organization & 1) This can cause delays in getting parts in. This could impact testing and integration schedules. & 1.) Having thorough internal design reviews with experienced members can minimize the number of recurring issues with manufacturing ensuring a streamlined manufacturing process 2.) Completing designs early and getting feedback from machine shops will allow the team to make edits to the design prior to design freezes. Open discussion with these shops will also allow members to gain exposure to best manufacturing practice. \\ \bottomrule
    \bottomrule
    \end{tabularx}
    }
\end{table}
