\chapter[Federal Aviation Regulations 14 CFR 107]{Federal Aviation Regulations 14 CFR, Part 107, Small Unmanned Aircraft Regulations}

The \acrfull{faa} rules for small unmanned aircraft (referred to as `\acrfullpl{uas}') operations other than model aircraft---Part 107 of \gls{faa} regulations--cover a broad spectrum of commercial and government uses for \acrshortpl{uav} weighing less than 55 pounds. This chapter lists the highlights of the rule.

\paragraph{Operating Requirements.} 
When you are manipulating the controls of a drone, always avoid manned aircraft and never operate in a careless or reckless manner. You must keep your drone within sight. Alternatively, if you use \gls{fpv} or similar technology, you must have a visual observer always keep your aircraft within unaided sight (for example, no binoculars). Neither you nor a visual observer can be responsible for more than one unmanned aircraft operation at a time.

You can fly during daylight (30 minutes before official sunrise to 30 minutes after official sunset, local time) or in twilight with appropriate anti-collision lighting. Minimum weather visibility is three miles from your control station. The maximum allowable altitude is 400 feet above the ground, higher if your drone remains within 400 feet of a structure. Maximum speed is 100 mph (87 knots).

You currently cannot fly a small \gls{uas} over anyone not directly participating in the operation, not under a covered structure, or not inside a covered stationary vehicle.  No operations from a moving vehicle are allowed unless you are flying over a sparsely populated area.

You can carry an external load if it is securely attached and does not adversely affect the flight characteristics or controllability of the aircraft. You also may transport property for compensation or hire within state boundaries provided the drone, including its attached systems, payload and cargo, weighs less than 55 pounds total and you obey the other flight rules. (Some exceptions apply to Hawaii and the District of Columbia.)

You can request a waiver of most restrictions if you can show your operation will provide a level of safety at least equivalent to the restriction from which you want the waiver.

\paragraph{Registration.}
Anyone flying under Part 107 has to register each drone they intend to operate. If your drone weighs less than \SI{55}{\poundm}, you can use the automated registration system.

\paragraph{Pilot Certification.}
To operate the controls of a small \gls{uas} under Part 107, you need a remote pilot certificate with a small \gls{uas} rating, or be under the direct supervision of a person who holds such a certificate

You must be at least 16 years old to qualify for a remote pilot certificate, and you can obtain it in one of two ways.

You may pass an initial aeronautical knowledge test at an \gls{faa}-approved knowledge testing center.

If you already have a Part 61 pilot certificate, you must have completed a flight review in the previous 24 months and you must take a small \gls{uas} online training course provided by the \gls{faa}.

If you have a Part 61 certificate, you will immediately receive a temporary remote pilot certificate when you apply for a permanent certificate. Other applicants will obtain a temporary remote pilot certificate upon successful completion of \gls{tsa} security vetting. We anticipate we will be able to issue temporary certificates within 10 business days after receiving a completed application.

\paragraph{\gls{uas} Certification.}
You are responsible for ensuring a drone is safe before flying, but the \gls{faa} does not require small \gls{uas} to comply with current agency airworthiness standards or obtain aircraft certification. For example, you will have to perform a preflight inspection that includes checking the communications link between the control station and the \gls{uas}.

\paragraph{Other Requirements.}
If you are acting as pilot in command, you have to comply with several other provisions of the rule:

\setlength{\parskip}{5.04pt}
\begin{itemize}
	\item {\fontsize{10pt}{12.0pt}\selectfont \textcolor[HTML]{333333}{You must make your drone available to the FAA for inspection or testing on request, and you must provide any associated records required to be kept under the rule.}\par}\par

	\item {\fontsize{10pt}{12.0pt}\selectfont \textcolor[HTML]{333333}{You must report any operation that results in serious injury, loss of consciousness, or property damage of at least \$500 to the FAA within 10 days}\par}
\end{itemize}\par

\paragraph{Waivers and Airspace Authorizations.}
The \gls{faa} can issue waivers to certain requirements of Part 107 if an operator demonstrates they can fly safely under the waiver without endangering other aircraft or people and property on the ground or in the air. Operations in Class G airspace are allowed without air traffic control permission. Operations in Class B, C, D and E airspace need \gls{atc} approval.

In November 2017, the \gls{faa} deployed the Low Altitude Authorization and Notification Capability (LAANC – pronounced ``LANCE'' ) for drone operators at several air traffic facilities in an evaluation to see how well the prototype system functions and to address any issues that arise during testing. A beta test expansion of the system began on April 30, 2018 to deploy LAANC incrementally at nearly 300 air traffic facilities covering approximately 500 airports. The final deployment will begin on September 13.

The \gls{faa} expects LAANC will ultimately provide near real-time processing of airspace authorization requests for drone operators nationwide. The system is designed to automatically approve most requests to operate in specific areas of airspace below designated altitudes.

