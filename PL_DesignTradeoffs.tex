\chapter{Payload Design Tradeoffs}

\section{Means of Securement}\label{PL:Tradeoffs:Securement}
	\subsection{Locking Mechanism}
		This is a filler paragraph
		
	\subsection{UAV Arm Configuration}
		\subsubsection{Parallel Unfolding Arms}
			This is a filler paragraph

		\subsubsection{Vertically Unfolding Arms}
			This is a filler paragraph

\section{Means of Deployment}\label{PL:Tradeoffs:Deployment}
	The challenge posed to the team regarding the release of the payload is to deploy the UAV in the proper orientation for flight. The UAV requires a fast, consistent, and reliable release mechanism to deploy it from the rocket. During descent, the UAV will be deployed out of the top of the rocket as described below. 

	\subsection{Launch Rails}
		The drone is secured within the rocket with guide rails. The retracted arms are connected to guide rails that hold the drone securely and will guide the drone out of the head of the rocket. The arms fold vertically as shown in the CAD models. Upon release, the drone will slide down the guide rails, out of the head, and the arms will flip open. This process must be controlled to ensure the drone comes out and deploys fully and in the correct orientation. A mechanism will be used to control the drones descent out of the rocket and ensure that the drone is ready before releasing it from the rocket.

	\subsection{Lowering Mechanism}
		To control the speed of the drone exiting the rocket, a lowering mechanism will be used to limit the speed. The mechanism can be either passively controlled through friction or actively controlled through a servo motor. The passive system relies on friction in the winch. 

		One option is for a servo motor to tactically lower the UAV by an exact amount at an exact speed. The UAV can be accelerated and decelerated at the beginning and ending of its descent to ensure it doesn't experience extreme jerk. This will ensure the further safety of the UAV. Since only the descent of the UAV needs to be controlled, the motor does not need to be powerful. The motor can be geared to the speed necessary to deploy the UAV. However, the motor can fail. It requires electronic input and has more points of failure than a passive friction system.
	
		The passive system will eliminate the points of failure that arise from the use of a servo. The winch can be built with a specific resistance to restrict the acceleration of the drone as gravity pulls it out of the rocket. One benefit of this solution is that it will also keep costs down. However, this solution poses a larger threat to the safety of the drone. The drone will accelerate out of the rocket and may be traveling at a high velocity when the limit of the winch is reached. In this case, the drone could experience a large jerk, which could cause damage.The descent will also have less control, resulting in a less consistent deployment.
	
	\subsection{Lowering Mechanism Release System}
		Once the UAV has been deployed outside of the rocket, the line attached to the winch must be disconnected. The team came up with three potential ways to accomplish this. The first option utilizes a sharp cutter attached to the UAV to sever the string. The second option uses a quick-release mechanism attached to the UAV to disconnect the line. Lastly, a nylon string could be attached to a winch and an electric connection could be used to burn through the nylon string. 

		The cutter mechanism would require mechanical components and a motor to execute the operation. The design is simple, but will require care to ensure that the cutter will completely sever the string in all conditions, such as in both the presence and the absence of tension. 
	
		The quick-release mechanism is similar to the cutter mechanism. It will require a motor or similar mechanical source to pull a pin. The advantage of this system is that it does not depend on the line being cut. This also allows the line to be made of almost any material, including options more durable than nylon that do not cut easily. However, the quick-release would need to be designed carefully to ensure that it is not accidentally triggered before intended. The risk of premature triggering of the release is significant, and factors heavily into the team’s consideration of the viability of this option. 
	
		The electric system is a viable option for burning through nylon string. This system has no moving parts, which corresponds to less points of failure. It is the most reliable option. However, it will could longer to burn through the string than it might take to execute the other options. Additionally, this system limits the material for the string to nylon. 
	

	\subsection{Nose Cone Jettison}
		\subsubsection{Without Parachute}
			This is a filler paragraph
		
		\subsubsection{With Parachute}
			This is a filler paragraph

\section{Payload Structures}\label{PL:Tradeoffs:Structures}
	\subsection{Landing Leg Design}
		\subsubsection{Retractable Landing Mechanism}
			This is a filler paragraph

		\subsubsection{Static Landing Mechanism}
			This is a filler paragraph

	\subsection{Ice Retrieval and Mobility Agent}
		\subsubsection{Pin Joint Mechanism}
			The goal of the payload is to retrieve ten milliliters of ice and move the ice from one bucket to another. To achieve this, the team decided to construct a claw-like mechanism to scoop the ice from underneath the drone, close the claw, and drop the ice in the bucket at the destination. The design has a servo rotate a bolt with a nut without moving the bolt vertically. The nut then moves with respect to the bolt, thus opening the claw as the nut moves down. To close the claw, the nut moves up and the lever arm becomes horizontal, thus pushing the top of claw arm outward. The bottom of the claw will have a slanted shape to easier scoop ice pieces and lower the risk of pushing the ice pieces outwards. 

			INSERT IMAGE HERE

		\subsubsection{Sliding Pin Mechanism}
			Another design the team is considering for the Ice Retrieval and Mobility Agent is a sliding pin mechanism. The sliding pin mechanism relies on the same movement of a nut sliding down a bolt to open and close the claws, but instead of a pin, it has slots that force the scooping mechanism to move out and in. To secure the scooping mechanism to the drone, two rods with pins at each end will be secured to the drone. The pin will connect the scooping mechanism to the rods and thus the drone. The figure shows the sliding pin mechanism closed (left) and open when the pin is near the top of the slot (right). 

			INSERT IMAGE HERE

		\subsubsection{Wormgear Mechanism}
			The third consideration was a design that operated using a worm gear to control the claw motion. A fixed position, vertically-oriented worm gear would be rotated using a servo. On either side of the worm gear are conventional round gears that naturally contra-rotate with respect to each other due to the radial symmetry of the worm gear action. The solid claw arms directly mount to their respective round gears, so that as the gears rotate, the arms swing open or closed about the same rotation axis as seen in INSERT FIGURE HERE. The rotation axes are provided by pin joints located on vertical support bars connected to the underside of the drone(not shown).

			It is easy to obtain the critical components of the design since worm gear packages are readily available in a wide range of sizes. Also, the design is very simplistic and easy to build. A major factor against the worm gear design is the cost of materials. Worm gears are harder to manufacture than round gears, and thus cost more. We decided against spending a large sum of money for a single mechanism. A critical aspect in the structural soundness of the design was identified in the region where the claw arms attached to the gears. The forces acting on the gears would have to be accounted for, and reinforcement of the region may offset the simplicity advantage of the overall design.
			
			INSERT IMAGE HERE.

\section{Payload Avionics}\label{PL:Tradeoffs:Avionics}
	\subsection{Computer Vision}
		\subsubsection{Algorithmic Approach}
			This is a filler paragraph

		\subsubsection{Deep Learning Approach}
			The deep learning approach to using computer vision on the drone is to utilize neural networks. Neural networks can be used to detect and classify objects; they are used in various emerging technologies, such as self-driving cars and automatic facial recognition. 
			Neural networks function by consecutively modeling small pieces of information and then combining them to form templates, and then finally weighing inputs with the templates to create a final prediction.
		
			During a training process, neural networks utilize algorithms that automatically find patterns in objects by evaluating external features of the object, such as color and structure, to divide objects into different classes. The process then updates weights, which correlate to the strength of association from an input image to the prediction, for the model and improves the performance every iteration. By doing so, it results in minimal error when evaluating the input images.
		
			For the competition, the process would require the team to feed the neural network images similar to that of the excavation area to “train” it to recognize what the excavation area would look like.These images would have labels and tags on them that identify each object. The neural network would analyze features such as the color of the boundary of the excavation area, and then update weights for the excavation area class. The same thing can be done with images of the “ice” sample that needs to be retrieved. Images with different amounts of ice can be input into the neural network to teach it the difference between areas with a high or low amount of ice concentration. Doing so would aid the drone in deciding the specific area it should retrieve the ice from.
		
			Neural networks have specific benefits and setbacks that need to be considered before choosing an approach to employ for computer vision. One useful benefit is that if the neural network is trained correctly, it can be extremely fast, and would be able to detect different objects accurately and reliably. On the other hand, a major setback to using a deep learning approach is the immense training time needed. The number of images needed to train the neural network to an adequate level is unknown, but could range from 1,000 to 100,000 comparison images, based on previous implementations.

	\subsection{Flight Controller}
		From the multitude of different flight controllers available that would suit the mission parameters, the team was able to narrow the list down to 3 choices. These flight controllers, the ArduPilot APM, the Pixhawk 4 and the Navio2, are all widely used by the UAV community. The team identified the factors that were most pertinent to the mission and weighted them by their importance. Ultimately, the flight controller in tandem with its companion computer would require the adequate processing power for the computer vision system and a high degree of reliability, which often varies depending upon software or physical design.

		\subsubsection{ArduPilot}
			The ArduPilot APM, although revolutionary on its initial release, has several downfalls. The most severe of which is its lackluster 8-bit processor. It consists of two components, the main processor board and the IMU shield that can be attached on top. The APM’s framework is based on that of the Arduino Mega and is capable of tasks such as autopilot, autonomous stabilization and way-point based navigation.

		\subsubsection{Pixhawk}
			The Pixhawk is also a renowned flight controller. It contains a much more powerful 32-bit processor and a built-in IMU + Barometer. One of the benefits of the Pixhawk is its ability to interface with a companion computer, such as a Raspberry Pi. A setup like this would be ideal for the high computing power a computer vision navigation system would require.

		\subsubsection{Navio2}
			The Navio2 is slightly different from the Pixhawk and ArduPilot APM; it requires a Raspberry Pi to function. The Navio2 itself is just a shield containing high resolution sensors, a UART interface and a GNSS receiver that fits onto a Raspberry Pi. This combined system coalesces to become the flight controller. A set up like this removes the need to carry an extra companion computer in addition to the flight controller because the flight controller system itself has a high power, 64-bit, 1.5 GHz processor on the Raspberry Pi which can handle the processing needs for the computer vision system.

		\begin{table}[H]
			\centering
			\caption{Flight Controller Fact Sheet}
			\label{tab: flight controller}
			\begin{tabularx}{1\linewidth}{X X X X}
				\toprule
				Factor & ArduPilot & Pixhawk + RaspberryPi & Navio2 \\
			   \midrule
			   	Size (mm) & 70.5 x 45 x 10 & 44 x 84 x 12 & 55 x 65 \\
				Mass (g) & 31.0 & 15.8 & 23.0 \\
				Ease of Integration & 8-bit processor requires aid of companion \mbox{computer} and several peripheral sensors & Requires configuration of linux files to set up UART bridge & Very easy due to extensively tested integration procedures\\
				Power \newline consumption (W) & 0.5 & 3.6 & 2.79 \\
				Reliability & No fall back option & Failures can occur in UART connection & Reliable integration due to 3rd-party product \mbox{design} \\
				\bottomrule
			\end{tabularx}
		\end{table}