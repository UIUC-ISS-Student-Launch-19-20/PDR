\chapter{Subscale}
\subsection{Overview}
A subscale rocket will be constructed to acquire data for analysis. The team decided to construct a half scale rocket, which will be 47.5 inches in length and 2 inches in diameter with a fairing up to 3 inches. The subscale rocket will weigh 73.6 ounces on the launch pad and reach an apogee of over 1,500 feet before descending with drogue and main parachutes.
    

 \subsection{Mass Statement}
\begin{tabularx}{0.8\linewidth}{X l}
\hline
\multicolumn{2}{*}{Mass Statement} \\
\hline
Component & Mass \\ \hline

\multirow{2}{*}{Nose Cone} & Nose Cone & 0.81350478 \\
 & \cellcolor{grey!25} Subtotal & \cellcolor{grey!25} 0.81350478 \\ \hline

\multirow{3}{*}{Fairing Tube} & Tube & .881848 \\
 & Payload & 5.51156 \\
 & \cellcolor{grey!25} Subtotal & \cellcolor{grey!25} 6.393408 \\ \hline

\multirow{6}{*}{Shoulder/Transition} & Shoulder & 0.49383488 \\
 & Avionics & 1.10231 \\
 & Bulkhead & .100751134 \\
 & Tube Coupler & 0.47178868 \\
 & Altimeter & 0.0062390746  \\
 & \cellcolor{grey!25} Subtotal & \cellcolor{grey!25} 2.174923769 \\ \hline

\multirow{4}{*}{Upper Body Tube} & Body Tube & 1.89376858 \\
 & Big Parachute & 0.4078547 \\
 & Shock Cord & 0.25794054 \\
 & \cellcolor{grey!25} Subtotal & \cellcolor{grey!25} 2.55956382 \\ \hline

\multirow{6}{*}{Switch Band and Coupler} & Switch Band & 0.135363668 \\
 & Tube Coupler & 0.5401319 \\
 & Bulkhead & 0.26234978 \\
 & Bulkhead & 0.26234978 \\
 & Coupler Avionics & 1.10231 \\
 & \cellcolor{grey!25} Subtotal & \cellcolor{grey!25} 2.302505128 \\ \hline

\multirow{12}{*}{Booster Tube} & Body Tube & 2.16493684 \\
 & Fins & 0.7495708 \\
 & Centering Rings & 0.186510852 \\
 & Drogue & 0.037037616 \\
 & Shock Cord & 0.25794054 \\
 & Inner Tube & 0.41226394 \\
 & Epoxy and Screws & 1.10231 \\
 & Launch Lug & .01 \\
 & Launch Lug & .01 \\
 & Motor & 6.46835508\\
 & (Motor Without Propellant) & (3.67289692) \\ \hline
 & \cellcolor{grey!25} Subtotal & \cellcolor{grey!25} 11.39892567 \\ \hline

Total Mass & 25.642831167 \\ \hline

\hline
\end{tabularx}

    \subsection{Motor Selection}
    The subscale will use the AeroTech H123W-14 motor to achieve a predicted apogee of 1,500 feet. The motor burns for 1.8 seconds, producing an average thrust of 27.7 pounds-force and a max thrust of 39.2 pounds-force at 0.3 seconds after ignition. Its thrust curve is shown in [FIGURE HERE].

    
    
    \subsection{Outer Airframe}
The outer booster and body tube airframes of the subscale rocket will be made of Blue Tube with a 2.0 inches outer diameter and the nose cone and fairing body tube will have a 3.0 inches diameter. The booster segment will be 16 inches long, providing ample space to store the drogue parachute, a shock cord, and the motor. The booster airframe will weigh 5.71 ounces, while the overall booster section will weigh 33.26 ounces. The body tube airframe will be 14 inches long, with the main parachute and another shock cord housed inside of it. This portion will weigh 8.604 ounces. The fairing transition length will be 3 inches as per the requirement that mandates that all coupler/airframe shoulders which are located at in-flight separation points will be at least 1 body diameter in length. This section weighs 1.95 ounces and will house the altimeter for the launch. The fairing body tube and nose cone have lengths of 4.5 inches and 9 inches, respectively, with their masses being 3.48 ounces and 4.65 ounces. This portion of the subscale launch vehicle will contain a 1 lb payload as a mass simulator.

    \subsection{Fins}
The subscale rocket will utilize three trapezoidal fins arranged in 120° offsets at the aft end of the rocket. The fins will be made from 0.125-inch plywood. Plywood was chosen over fiberglass because less force will act on the fins during the subscale flight, and cost benefits can be realized since the team has the ability to manufacture plywood fins via laser cutting.

    \subsection{Avionics}
The subscale rocket will have an avionics bay located in the coupler segment of the rocket. On board the avionics bay will be an altimeter to document flight events and apogee of the subscale rocket. The model of altimeter will be a StratoLogger CF.
The team will not be using the StratoLogger to set ejection charges. Instead, the subscale rocket will utilize the motor ejection charge on the motor for separation and parachute deployment. This approach greatly reduces the complexity of the avionics bay and avoids wiring, which results in greater reliability. The motor ejection charge of the H97J-10 motor is fired shortly after apogee, which conveniently allows for the implementation of this method.

    \subsection{Payload}
 A mass simulator of 1 pound was estimated to best imitate the drone payload that will be present in the full-scale vehicle. This weight results in a similar mass distribution to that of the full-scale, which allows for a more accurate subscale test flight. 

The team will not be launching a subscale drone paired with the subscale rocket. It would not be feasible to scale down the drone as well as deploy it mid-flight. Instead, an independent test will be conducted on the full-scale drone with a 6-inch Blue Tube segment.
   
    \subsection{Recovery}
The subscale rocket will descend under a two parachute system like its full-scale counterpart. A drogue parachute with 12 inch diameter will be deployed approximately 1 second after apogee. The launch vehicle will separate into two components at the coupler via ejection charge but will stay connected with a shock cord. At 500 feet above ground level the main parachute, with diameter 48 inches, will deploy via ejection charge and will then separate the rocket once again at the upper end of the coupler and will also stay attached via shock cord. The ground hit velocity has been simulated to be 18.5 feet per second, and the descent from apogee is 39.9 seconds. Both of these simulations were conducted in OpenRocket. 

    \subsection{Simulations}
    
