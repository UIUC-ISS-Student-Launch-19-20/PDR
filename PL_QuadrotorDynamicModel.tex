\chapter{Quadrotor Dynamic Model}

\section{Coordinate Systems}

Let us consider the following four coordinate systems: \begin{enumerate*}[label=(\roman*), noitemsep]
	\item inertial,
	\item Earth-fixed,
	\item instantaneous topocentric, and
	\item body-fixed
\end{enumerate*} \citep[pp.~3\textit{f.}]{Singh1975}. For these systems, the axes are defined as follows:

\textit{Inertial system ($\text{O}$ at Earth center)}

\begin{itemize}[noitemsep]
	\item[\textbf{X}] On the Earth equatorial plane, pointing to the zero longitude at take-off
	\item[\textbf{Y}] On the Earth equatorial plane, pointing to the \SI{90}{\degree} longitude at take-off
	\item[\textbf{Z}] Perpendicular to the equatorial plane, pointing to the North Pole
\end{itemize}

\textit{Earth-fixed system ($\text{O}_\text{E}$ at Earth center)}

\begin{itemize}[noitemsep]
	\item[$\mathbf{X_E}$] On the Earth equatorial plane, always pointing to the Greenwich (\SI{0}{\degree}) longitude
	\item[$\mathbf{Y_E}$] On the Earth equatorial plane, always pointing to the \SI{90}{\degree} longitude
	\item[$\mathbf{Z_E}$] Perpendicular to the equatorial plane, pointing to the North Pole
\end{itemize}

\textit{Instantaneous topocentric system ($\text{O}_\text{T}$ at the projection point of the moving \gls{uav} on the Earth surface)}

\begin{itemize}[noitemsep]
	\item[$\mathbf{x_T}$] On the local horizon plane tangent to the instantaneous projection point of the \gls{uav}, directed along the local geocentric North
	\item[$\mathbf{y_T}$] On the local horizon plane tangent to the instantaneous projection point of the \gls{uav}, directed along the local geocentric North
	\item[$\mathbf{z_T}$] Perpendicular to the instantaneous tangent plane, directed along the geocentric radius vector and pointing toward the Earth center
\end{itemize}

\textit{Body-fixed system ($\text{O}_\text{B}$ at the center of gravity of the \gls{uav})}

\begin{itemize}[noitemsep]
	\item[$\mathbf{x_B}$] Along the \gls{uav} principle (longitudinal) axis, positive forward
	\item[$\mathbf{y_B}$] Normal to the $x_B$--$z_B$ symmetric plane, completing the right-hand system
	\item[$\mathbf{z_B}$] In the principle plane of symmetry of the \gls{uav}, perpendicular to the $x_B$ axis and positive downward
\end{itemize}

We will chiefly confine our discussion to the $O_T$ (instantaneous topocentric system) and $O_B$ (body-fixed) frames, as the distances the vehicle will travel allow for a flat Earth approximation with constant uniform gravity \citep{Burko2005}. Let us now consider the coordinate transformation between these two frames:

\subsection{Coordinate Transformation}

Let us define the following Euler angles:

\begin{itemize}[noitemsep]
	\item[$\boldsymbol{\theta}$] Pitch
	\item[$\boldsymbol{\psi}$] Yaw
	\item[$\boldsymbol{\phi}$] Roll
\end{itemize}

We then find the transformation from the instantaneous topocentric frame to the body-centric frame to be \citep[Eq.~2-2]{Kurak2018}:

\begin{equation}
\begin{split}
	[R]_{T \rightarrow B} (\theta, \psi, \phi) &=
	\begin{bmatrix}
		1 & 0 & 0 \\
		0 & \cos\phi & \sin\phi \\
		0 & -\sin\phi & \cos\phi
	\end{bmatrix}
	\begin{bmatrix}
		\cos\theta & 0 & -\sin\theta \\
		0 & 1 & 0 \\
		\sin\theta & 0 & \cos\theta
	\end{bmatrix}
	\begin{bmatrix}
		\cos\psi & \sin\psi & 0 \\
		-\sin\psi & \cos\psi & 0 \\
		0 & 0 & 1
	\end{bmatrix} \\
	&= 
	\begin{bmatrix}
		\cos\theta \cos\psi & \cos\theta \sin\psi & -\sin\theta \\
			\sin\theta \cos\psi \sin\phi - \sin\psi \cos\phi & \sin\theta \sin\psi \sin\phi + \cos\psi \cos\phi & \cos\theta \sin\phi \\
			\sin\theta \cos\psi \cos\phi + \sin\psi \sin\phi & \sin\theta \sin\psi \cos\phi - \cos\psi \sin\phi & \cos\theta \cos\phi \\
	\end{bmatrix}
\end{split}
\end{equation}

Given the nature of the transformation matrix, it is readily found that:

\begin{equation}
	[R]_{B \rightarrow T} = [R]_{T \rightarrow B}^{-1} = [R]_{T \rightarrow B}^\intercal
\end{equation}

\section{Dynamic Model}

Given the complex nature of the rotor-based \gls{uav}, we will make the following a priori assumptions to aid in the derivation of the vehicle model:

\begin{enumerate}[noitemsep]
	\item The \gls{uav} is a rigid body.
	\item The \gls{toi} of the \gls{uav} is approximated as the \gls{moi} of several objects.
	\item The \gls{cm} coincides with the \gls{uav}'s geometrical centroid.
	\item The \gls{moi} of the propellers is neglected.
	\item Time delay of commands is neglected.
	\item Aerodynamic drag force is neglected.
\end{enumerate}

While most of the aforestated simplifications are intuitively sound, neglecting the aerodynamic drag force begs for justification. \citet{Castillo2017} provide the following approximation to the \gls{uav} drag force:

\begin{equation}
	\vec{f}_{d_k} = C_{D_k} \rho A_k V_k (V_{w_k} - V_k) \hat{\vec{k}}, \quad k: x_b, y_b, z_b 
\end{equation} 

Where $\hat{\vec{k}}$ is the unit vector in $\vec{k}$ direction, $A_k$ is the reference area by which the drag coefficient $C_{D_k}$ is determined in the $\vec{k}$-direction. $V_k$ and $V_{w_k}$ are the vehicle velocity and wind velocity in $O_B$, respectively, and $\rho$ is the air density, which is assumed to be constant. In this formulation, $C_{D_k}$ is to be determined through simulation (\gls{cfd}) or experiment. However, this poses a significant difficulty: the \gls{uav} consists of many distinct components with orientation and speed dependent drag characteristics, making the definition of a constant $C_{D_k}$ inadvisable. As a matter of fact, \citet{Castillo2017} treat the drag force as an unknown disturbance that is to be counteracted by the control system, thus warranting the term to be dropped.

Let us define $\vec{T}$ and $\vec{H}$, the thrust force and hub torque for every motor--propeller system, respectively. Research shows that these forces scale with the square of the angular rate of the propellor \citep{Kurak2018, Castillo2017, Winslow2018, Bershadsky2016, Luukkonen2011}, i.e.:

\begin{align}
	T_i &= k_T \omega_i^2 \\
	H_i &= k_H \omega_i^2
\end{align}

where $i$ denotes the $i$'th propeller, and we assume the propellers to be identical, yielding constant $k_T, k_H$. These coefficients can be found experimentally. Following the Newton--Euler approach presented by \citet{Kurak2018}, we find for a symmetric four-propeller \gls{uav} layout:

\begin{align}
	\ddot\phi &= \frac{\ell (T_2 - T_4) - (I_z - I_y) \dot\theta \dot\psi}{I_x} \\
	\ddot\theta &= \frac{\ell (T_3 - T_1) - (I_x - I_z) \dot\phi \dot\psi}{I_y} \\
	\ddot\psi &= \frac{(H_1 + H_3) - (H_2 + H_4) - (I_y - I_x)\dot\phi \dot\theta}{I_z} \\
	\ddot x &= \frac{(\cos\phi \sin\theta \cos\psi + \sin\phi \sin\psi) \sum_{i=1}^{4} T_i}{m} \\
	\ddot y &= \frac{(\cos\phi \sin\theta \sin\psi - \sin\phi \cos\psi) \sum_{i=1}^{4} T_i}{m} \\
	\ddot z &= \frac{(\cos\phi \cos\theta) \sum_{i=1}^{4} T_i - m g}{m}
\end{align}

where $I_k = I_{kk}, k:x,y,z$ are the diagonal elements of the \gls{toi} $\vec{I}$, $\ell$ is the $L_2$-distance between the propeller and the origin of the $O_B$-frame ($\ell$ is constant to satisfy symmetry).

\subsection{Nonlinear state space model}

Let us define the following state variable:

\begin{equation}
	\begin{split}
		\vec{x} = 
		\begin{bmatrix}
			\phi \\
			\dot\phi \\
			\theta \\
			\dot\theta \\
			\psi \\
			\dot\psi \\
			z \\
			\dot z \\
			x \\
			\dot x \\
			y \\
			\dot y
		\end{bmatrix}
		=
		\begin{bmatrix}
			x_1 \\
			x_2 = \dot{x}_1 \\
			x_3 \\
			x_4 = \dot{x}_3 \\
			x_5 \\
			x_6 = \dot{x}_5 \\
			x_7 \\
			x_8 = \dot{x}_7 \\
			x_9 \\
			x_{10} = \dot{x}_9 \\
			x_{11} \\
			x_{12} = \dot{x}_{11} \\
		\end{bmatrix}
	\end{split}
\end{equation}

The force control input is then defined as:

\begin{equation}
	\vec{u}^* =
	\begin{bmatrix}
		\sum_{i=1}^{4} T_i \\
		T_2 - T_4 \\
		T_3 - T_1 \\
		(H_1 + H_3) - (H_2 + H_4)
	\end{bmatrix}
\end{equation}

Defining the angular rate control input as:

\begin{equation}
	\vec{u} = 
	\begin{bmatrix}
		\omega_1^2 \\ \omega_2^2 \\ \omega_3^2 \\ \omega_4^2 \\
	\end{bmatrix}
\end{equation}

Using this definition, we find the force control input $\vec{u}^*$ to be related to the angular rate control input $\vec{u}$ as follows:

\begin{equation}
	\vec{u}^* =
	\begin{bmatrix}
		k_T & k_T & k_T & k_T \\
		0 & k_T & 0 & - k_T \\
		- k_T & 0 & k_T & 0 \\
		k_H & - k_H & k_H & - k_H
	\end{bmatrix}
	\vec{u}
\end{equation}

Let us now define the following constants:

\begin{align}
	a_1 &= \frac{I_y - I_z}{I_x} && a_2 = \frac{I_z - I_x}{I_y} && a_3 = \frac{I_x - I_y}{I_z} \\
	b_1 &= \frac{\ell}{I_x} &&b_2 = \frac{\ell}{I_y} &&b_3 = \frac{\ell}{I_3}
\end{align}

and the following rotations:

\begin{equation}\label{eq:rotations}
\begin{split}
	r_t &= \cos\phi \cos\theta \\
	r_x &= \cos\phi \sin\theta \cos\psi + \sin\phi \sin\psi \\
	r_y &= \cos\phi \sin\theta \sin\psi - \sin\phi \cos\psi 
\end{split}
\end{equation}

The nonlinear state space formulation is then found to be:

\begin{equation}\label{eq:nonlinearss}
	\dot{\vec{x}} =
	\begin{bmatrix}
		x_2 \\
		x_4 x_6 a_1 + b_1 u^*_2 \\
		x_4 \\
		x_2 x_6 a_2 + b_2 u^*_3 \\
		x_6 \\
		x_2 x_4 a_3 + b_3 u^*_4 \\
		x_8 \\
		-g + r_t(x_1, x_3) u^*_1 / m \\
		x_{10} \\
		r_x(x_1, x_3, x_5) u^*_1 / m \\
		x_{12} \\
		r_y(x_1, x_3, x_5) u^*_1 / m 
	\end{bmatrix}
\end{equation}

\subsection{Linearized state space model}

Let us apply a small angle approximation on Eq.~\ref{eq:rotations}, giving:

\begin{equation}\label{eq:rotationsSAM}
\begin{split}
	r_t &\approx 1 \\
	r_x &\approx \theta \\
	r_y &\approx \theta\psi - \phi \approx -\phi
\end{split}
\end{equation}

To linearize Eq.~\ref{eq:nonlinearss}, we must find a suitable equilibrium point, where $\dot{\vec{x}} = \vec{0}$. This holds for:

\begin{equation}
\begin{split}
	\bar{\vec{x}} &= \begin{bmatrix}
		0 & 0 & 0 & 0 & 0 & 0 & \bar{x}_7 & 0 & \bar{x}_9 & 0 & \bar{x}_{11}
	\end{bmatrix}^\intercal \\
	\bar{\vec{u}}^* &= 
	\begin{bmatrix}
		mg & 0 & 0 & 0
	\end{bmatrix}^\intercal
\end{split}
\end{equation}

Linearizing Eq.~\ref{eq:nonlinearss} about $(\bar{\vec{x}}, \bar{\vec{u}}^*)$, we obtain:

\begin{equation}
\begin{split}
	\vec{A} &= \eval[1]{\pd{\dot{\vec{x}}(\vec{x},\vec{u}^*)}{\vec{x}}}_{(\bar{\vec{x}}, \bar{\vec{u}}^*)} \\
	\vec{B} &= \eval[1]{\pd{\dot{\vec{x}}(\vec{x},\vec{u}^*)}{{\vec{u}^*}}}_{(\bar{\vec{x}}, \bar{\vec{u}}^*)}
\end{split}
\end{equation}

which yields the following state space formulation:

\begin{equation} \label{eq:linearss}
\begin{split}
	\dot{\vec{x}} &=
	\underbrace{
	\begin{bmatrix}
		0 & 1 & 0 & 0 & 0 & 0 & 0 & 0 & 0 & 0 & 0 & 0 \\
		0 & 0 & 0 & 0 & 0 & 0 & 0 & 0 & 0 & 0 & 0 & 0 \\
		0 & 0 & 0 & 1 & 0 & 0 & 0 & 0 & 0 & 0 & 0 & 0 \\
		0 & 0 & 0 & 0 & 0 & 0 & 0 & 0 & 0 & 0 & 0 & 0 \\
		0 & 0 & 0 & 0 & 0 & 1 & 0 & 0 & 0 & 0 & 0 & 0 \\
		0 & 0 & 0 & 0 & 0 & 0 & 0 & 0 & 0 & 0 & 0 & 0 \\
		0 & 0 & 0 & 0 & 0 & 0 & 0 & 1 & 0 & 0 & 0 & 0 \\
		0 & 0 & 0 & 0 & 0 & 0 & 0 & 0 & 0 & 0 & 0 & 0 \\
		0 & 0 & 0 & 0 & 0 & 0 & 0 & 0 & 0 & 1 & 0 & 0 \\
		0 & g & 0 & 0 & 0 & 0 & 0 & 0 & 0 & 0 & 0 & 0 \\
		0 & 0 & 0 & 0 & 0 & 0 & 0 & 0 & 0 & 0 & 0 & 1 \\
		-g & 0 & 0 & 0 & 0 & 0 & 0 & 0 & 0 & 0 & 0 & 0
	\end{bmatrix}
	}_{\vec{A}}
	\vec{x}
	+
	\underbrace{
	\begin{bmatrix}
		0 & 0 & 0 & 0 \\
		0 & b_1 & 0 & 0 \\
		0 & 0 & 0 & 0 \\
		0 & 0 & b_2 & 0 \\
		0 & 0 & 0 & 0 \\
		0 & 0 & 0 & b_3 \\
		0 & 0 & 0 & 0 \\
		1/m & 0 & 0 & 0 \\
		0 & 0 & 0 & 0 \\
		0 & 0 & 0 & 0 \\
		0 & 0 & 0 & 0 \\
		0 & 0 & 0 & 0
	\end{bmatrix}
	}_{\vec{B}^*}
	\vec{u}^* \\
	&= \vec{A} \vec{x}
	+
	\underbrace{
	\begin{bmatrix}
		0 & 0 & 0 & 0 \\
		0 & b_1 k_T & 0 & -b_1 k_T \\
		0 & 0 & 0 & 0 \\
		-b_2 k_T & 0 & b_2 k_T & 0 \\
		0 & 0 & 0 & 0 \\
		b_3 k_H & -b_3 k_H & b_3 k_H & -b_3 k_H \\
		0 & 0 & 0 & 0 \\
		\frac{k_T}{m} & \frac{k_T}{m} & \frac{k_T}{m} & \frac{k_T}{m} \\
		0 & 0 & 0 & 0 \\
		0 & 0 & 0 & 0 \\
		0 & 0 & 0 & 0 \\
		0 & 0 & 0 & 0 \\
	\end{bmatrix}
	}_{\vec{B}}
	\vec{u}
\end{split}
\end{equation}

\subsection{Discretization}

\gls{zoh} discretization of Eq.~\ref{eq:linearss}, gives for constant sampling time $h$:

\begin{equation}
\begin{split}
	\vec{\Phi} &= e^{\vec{A} h} = 
	\begin{bmatrix}
		1 & h & 0 & 0 & 0 & 0 & 0 & 0 & 0 & 0 & 0 & 0 \\
		0 & 1 & 0 & 0 & 0 & 0 & 0 & 0 & 0 & 0 & 0 & 0 \\
		0 & 0 & 1 & h & 0 & 0 & 0 & 0 & 0 & 0 & 0 & 0 \\
		0 & 0 & 0 & 1 & 0 & 0 & 0 & 0 & 0 & 0 & 0 & 0 \\
		0 & 0 & 0 & 0 & 1 & h & 0 & 0 & 0 & 0 & 0 & 0 \\
		0 & 0 & 0 & 0 & 0 & 1 & 0 & 0 & 0 & 0 & 0 & 0 \\
		0 & 0 & 0 & 0 & 0 & 0 & 1 & h & 0 & 0 & 0 & 0 \\
		0 & 0 & 0 & 0 & 0 & 0 & 0 & 1 & 0 & 0 & 0 & 0 \\
		0 & \frac{g h^2}{2} & 0 & 0 & 0 & 0 & 0 & 0 & 1 & h & 0 & 0 \\
		0 & g h & 0 & 0 & 0 & 0 & 0 & 0 & 0 & 1 & 0 & 0 \\
		-\frac{g h^2}{2} & -\frac{g h^3}{6} & 0 & 0 & 0 & 0 & 0 & 0 & 0 & 0 & 1 & h \\
		-g h & -\frac{g h^2}{2} & 0 & 0 & 0 & 0 & 0 & 0 & 0 & 0 & 0 & 1 \\
	\end{bmatrix} \\
	\vec{\Gamma} &= \int_0^h e^{\vec{A} s} \dif s \cdot \vec{B} =
	\begin{bmatrix}
		0 & \frac{1}{2} b_1 h^2 k_T & 0 & -\frac{1}{2} b_1 h^2 k_T \\
		0 & b_1 h k_T & 0 & b_1 (-h) k_T \\
		-\frac{1}{2} b_2 h^2 k_T & 0 & \frac{1}{2} b_2 h^2 k_T & 0 \\
		b_2 (-h) k_T & 0 & b_2 h k_T & 0 \\
		\frac{1}{2} b_3 h^2 k_H & -\frac{1}{2} b_3 h^2 k_H & \frac{1}{2} b_3 h^2 k_H &
		-\frac{1}{2} b_3 h^2 k_H \\
		b_3 h k_H & b_3 (-h) k_H & b_3 h k_H & b_3 (-h) k_H \\
		\frac{h^2 k_T}{2 m} & \frac{h^2 k_T}{2 m} & \frac{h^2 k_T}{2 m} & \frac{h^2 k_T}{2 m} \\
		\frac{h k_T}{m} & \frac{h k_T}{m} & \frac{h k_T}{m} & \frac{h k_T}{m} \\
		0 & \frac{1}{6} b_1 g h^3 k_T & 0 & -\frac{1}{6} b_1 g h^3 k_T \\
		0 & \frac{1}{2} b_1 g h^2 k_T & 0 & -\frac{1}{2} b_1 g h^2 k_T \\
		0 & -\frac{1}{24} b_1 g h^4 k_T & 0 & \frac{1}{24} b_1 g h^4 k_T \\
		0 & -\frac{1}{6} b_1 g h^3 k_T & 0 & \frac{1}{6} b_1 g h^3 k_T \\
	\end{bmatrix}
\end{split}
\end{equation}

with the following state space system:

\begin{equation}
	\vec{x}(k+1) = \vec{\Phi} \vec{x}(k) + \vec{\Gamma}\vec{u}(k)
\end{equation}

\section{Simulation}

From \citet{Tayebi2004}, we adopt the following values for our simulation:

\begin{table}[H]
	\centering
	\caption[Simulation parameters]{Simulation parameters \cite{Tayebi2004}}
	\label{tab:tayebiparams}
	\begin{tabularx}{.4\linewidth}{l X}
		\toprule
		\textbf{Parameter} & \textbf{Value} \\
		\midrule
		$g$ & \SI{9.81}{\meter\per\second\squared} \\
		$m$ & \SI{0.468}{\kilo\gram} \\
		$\ell$ & \SI{0.225}{\meter} \\
		$k_T$ & \SI{2.980e-6}{(\radian\squared\per\second\squared)\per\newton} \\
		$k_H$ & \SI{1.140e-7}{(\radian\squared\per\second\squared)\per\newton\meter} \\
		$I_x$ & \SI{4.856e-3}{\kilo\gram\per\meter\squared} \\
		$I_y$ & \SI{4.856e-3}{\kilo\gram\per\meter\squared} \\
		$I_z$ & \SI{8.801e-3}{\kilo\gram\per\meter\squared} \\
		\bottomrule \\
	\end{tabularx}
\end{table}

Letting $h = \SI{1e-3}{\second}$, we obtain:

\begin{equation}
\begin{split}
	\vec{\Phi} &= 
	\begin{bmatrix}
		1 & 0.001 & 0 & 0 & 0 & 0 & 0 & 0 & 0 & 0 & 0 & 0 \\
		0 & 1 & 0 & 0 & 0 & 0 & 0 & 0 & 0 & 0 & 0 & 0 \\
		0 & 0 & 1 & 0.001 & 0 & 0 & 0 & 0 & 0 & 0 & 0 & 0 \\
		0 & 0 & 0 & 1 & 0 & 0 & 0 & 0 & 0 & 0 & 0 & 0 \\
		0 & 0 & 0 & 0 & 1 & 0.001 & 0 & 0 & 0 & 0 & 0 & 0 \\
		0 & 0 & 0 & 0 & 0 & 1 & 0 & 0 & 0 & 0 & 0 & 0 \\
		0 & 0 & 0 & 0 & 0 & 0 & 1 & 0.001 & 0 & 0 & 0 & 0 \\
		0 & 0 & 0 & 0 & 0 & 0 & 0 & 1 & 0 & 0 & 0 & 0 \\
		0 & 4.905\times10^{-6} & 0 & 0 & 0 & 0 & 0 & 0 & 1 & 0.001 & 0 & 0
		\\
		0 & 0.00981 & 0 & 0 & 0 & 0 & 0 & 0 & 0 & 1 & 0 & 0 \\
		-4.905\times10^{-6} & -1.635\times10^{-9} & 0
		& 0 & 0 & 0 & 0 & 0 & 0 & 0 & 1 & 0.001 \\
		-0.00981 & -4.905\times10^{-6} & 0 & 0 & 0 & 0 & 0 & 0 & 0 & 0 & 0
		& 1 \\
	\end{bmatrix} \\
	\vec{\Gamma} &= 
	\begin{bmatrix}
		1 & 0.001 & 0 & 0 & 0 & 0 & 0 & 0 & 0 & 0 & 0 & 0 \\
		0 & 1 & 0 & 0 & 0 & 0 & 0 & 0 & 0 & 0 & 0 & 0 \\
		0 & 0 & 1 & 0.001 & 0 & 0 & 0 & 0 & 0 & 0 & 0 & 0 \\
		0 & 0 & 0 & 1 & 0 & 0 & 0 & 0 & 0 & 0 & 0 & 0 \\
		0 & 0 & 0 & 0 & 1 & 0.001 & 0 & 0 & 0 & 0 & 0 & 0 \\
		0 & 0 & 0 & 0 & 0 & 1 & 0 & 0 & 0 & 0 & 0 & 0 \\
		0 & 0 & 0 & 0 & 0 & 0 & 1 & 0.001 & 0 & 0 & 0 & 0 \\
		0 & 0 & 0 & 0 & 0 & 0 & 0 & 1 & 0 & 0 & 0 & 0 \\
		0 & 4.905\times10^{-6} & 0 & 0 & 0 & 0 & 0 & 0 & 1 & 0.001 & 0 & 0
		\\
		0 & 0.00981 & 0 & 0 & 0 & 0 & 0 & 0 & 0 & 1 & 0 & 0 \\
		-4.905\times10^{-6} & -1.635\times10^{-9} & 0
		& 0 & 0 & 0 & 0 & 0 & 0 & 0 & 1 & 0.001 \\
		-0.00981 & -4.905\times10^{-6} & 0 & 0 & 0 & 0 & 0 & 0 & 0 & 0 & 0
		& 1 \\
	\end{bmatrix} 
\end{split}
\end{equation}

\section{Controller Design}

This section details the controller setup and design for the \gls{uav}. We concern ourself chiefly with the servo problem, with path generation being delegated to the guidance law. Given the nature of the problem, we are forced to employ the \gls{lqr} methodology to tune the gains for this \gls{mimo} system.
	
\subsection{\gls{lqr} Tuning}

Following the guidelines from \citep{Franklin2002}, we are to work with the following cost function:

\begin{equation}
	\mathcal{J} = \rho \vec{x}^\intercal \vec{H}_w^\intercal \overline{\vec{Q}}_1 \vec{H}_w \vec{x} + \vec{u}^\intercal \vec{Q}_2 \vec{u}
\end{equation}

where $\vec{Q}$ is a diagonal matrix containing elements that equal the inverse of the square of the maximum deviation of the states/inputs of interest. The value of $\rho$ is to be tuned by trial and error, considering the properties of the response.

\paragraph{State weighting matrix.} Since we wish to have full state control with minimal deviations on every state, we will pose $\vec{H}_w = \vec{I}_{12}$. Given this choice of $\vec{H}_w$, we find $\overline{\vec{Q}}_1 = \vec{Q}_1$. We find $\overline{\vec{Q}}_1$ to be of the form:

\begin{equation}
	\overline{\vec{Q}}_1 =
	\begin{bmatrix}
		\Delta x_1^{-2} & 0 & \cdots & 0 \\
		0 & \Delta x_2^{-2} & \ddots & \vdots \\
		\vdots & \ddots & \ddots & 0 \\
		0 & \cdots & 0 & \Delta x_{12}^{-2} 
	\end{bmatrix}
\end{equation}

where the entries are tabulated in Tab.~\ref{tab:maximum state deviation}. The rationale behind the magnitude of these bounds lies in the mission objective of the \gls{uav}; we wish to accomplish minute maneuvering and near-stationary hovering with closely matching attitude and small slew rate. As can be seen from the values, we place a greater importance on the vertical ($z$-axis) motion, such that the vehicle can accomplish accurate terrain following and close-quarters operations close to the surface.

\begin{table}[H]
	\centering
	\caption{Maximum state deviation values for use in the $\overline{\vec{Q}}_1$ matrix in the \gls{lqr} routine}
	\label{tab:maximum state deviation}
	\begin{tabularx}{.55\linewidth}{l l X}
		\toprule
		\textbf{State} & \textbf{Variable} & \textbf{Maximum state deviation} \\
		\midrule
		$\Delta x_1$ & $\phi$ & $\SI{2.5}{\degree} \approx \SI{4.36e-2}{\radian}$ \\
		$\Delta x_2$ & $\dot\phi$ & $\SI{1.25}{\degree\per\second} \approx \SI{2.18e-2}{\radian\per\second}$ \\
		$\Delta x_3$ & $\theta$ & $\SI{2.5}{\degree} \approx \SI{4.36e-2}{\radian}$ \\
		$\Delta x_4$ & $\dot\theta$ & $\SI{1.25}{\degree\per\second} \approx \SI{2.18e-2}{\radian\per\second}$ \\
		$\Delta x_5$ & $\psi$ & $\SI{5}{\degree} \approx \SI{8.73e-2}{\radian}$ \\
		$\Delta x_6$ & $\dot\psi$ & $\SI{2.5}{\degree\per\second} \approx \SI{4.36e-2}{\radian\per\second}$ \\
		$\Delta x_7$ & $z$ & $\SI{0.025}{\meter}$ \\
		$\Delta x_8$ & $\dot z$ & $\SI{0.0125}{\meter\per\second}$ \\
		$\Delta x_9$ & $x$ & $\SI{0.05}{\meter}$ \\
		$\Delta x_{10}$ & $\dot x$ & $\SI{0.025}{\meter\per\second}$ \\
		$\Delta x_{11}$ & $y$ & $\SI{0.05}{\meter}$ \\
		$\Delta x_{12}$ & $\dot y$ & $\SI{0.025}{\meter\per\second}$ \\
		\bottomrule
	\end{tabularx}
\end{table}

\paragraph{Control weighting matrix.} For the control input, we wish to minimize the use of excessive throttle. We can accomplish this by setting $\Delta u_i = 2^{-1/2} \omega^2_{\text{max}}$ for $i \in [1,4]$. This will translate to a maximum of $\pm 71\%$ throttle on each rotor, yielding a thrust cap at the \gls{rms} value of the maximum thrust, $T_{\text{cap}} = T_{\text{max}}/\sqrt{2}$, if we assume it to vary as a sinusoid. Similar to the state weighting matrix, the control weighting matrix will assume the form of:

\begin{equation}
	\begin{bmatrix}
		\Delta u_1^{-2} & 0 & \cdots & 0 \\
		0 & \Delta u_2^{-2} & \ddots & \vdots \\
		\vdots & \ddots & \ddots & 0 \\
		0 & \cdots & 0 & \Delta u_{4}^{-2} 
	\end{bmatrix}
\end{equation}

Summarizing, we obtain the following cost function:

\begin{equation}
	\mathcal{J} = \rho \vec{x}^\intercal \vec{Q} \vec{x} + \vec{u}^\intercal \vec{Q}_2 \vec{u}
\end{equation}