\chapter{Leading Design of the Payload}

	\section{Deployment}\label{PL:Design:Deployment}
		The challenge posed to the team regarding the release of the payload is to deploy the UAV in the proper orientation for flight. The UAV requires a fast, consistent, and reliable release mechanism to deploy it from the rocket. During descent, the UAV will be deployed out of the top of the rocket as described below. 
		\subsection{Guide Rails}
			The UAV will be secured to the rocket by a system of guide rails. These rails will be connected to both the base of the fairing section and the retracted arms of the UAV. As the UAV is lowered out of the rocket, the arms will be released into the open position and ready for operation, pending approval from the RSO. The lowering of the UAV will be described by the winch system below.  

			\subsection{Lowering Mechanism}
				To lower the UAV out of the rocket body, the team has chosen to use a servo motor to power the winch. The team found this option suitable because it will allow the drone to be lowered at a constant speed within the safety margin. The team weighed the success of the mission over the complexities involved with integrating the servo powered winch system. In the case of electrical failure relating to the winch’s servo, the UAV will be lowered by gravity assisted by mechanical resistance from the dead servo. Leveraging the mechanical resistance, the servo-winch system will be within the safety margin and able to operate in the case of servo failure.

			\subsection{Lowering Mechanism Release System}
				To sever the winch line from the UAV, the team selected the method of burning a nylon string using an electric system. The other options were eliminated due to the complexity of the mechanical execution of the concepts; that is to say, the team feared that too many moving parts would lead to an overly complex release mechanism. The simplicity of severing a nylon string via heat appealed to the team because of its small risk of failure. The team plans on performing several rounds of tests to ensure that the electric system will be able to sever the nylon cord under any circumstances. 


	\section{Drone Structures}\label{PL:Design:Structures}
		\subsection{Arm Configuration}
			The drone will have four arms that rotate vertically such that the arms of the drone fit inside the fairing of the upper body tube. One other drone configuration was considered where the drone arms rotated sideways into itself, however this design was rejected primarily due to the inability for this configuration to fit inside the fairing. As a result, the vertical swinging arms chosen for the drone. These arms are placed on top of a hinge on the base of the drone with "bushings" placed on the ends. These bushings will align with aluminum poles connected to the base of the fairing which will make up the deployment system, which is described in another section.

		\subsection{Landing Legs}
			The drone will additionally have landing legs that are attached mechanically to the arms of the drone. Each leg will be attached along the arm close to the rotors. Another bar will be attached from the base plate of the drone, connecting to the leg, forming a four-bar linkage. This allows the legs of the drone to forever be perpendicular to the ground no matter the direction that the arm faces. This also allows them to fold up into the body tube without the need for additional servo motors or electrical components. Another design of static legs attached to the base plate of the drone was considered, but the distance between legs were so small that the drone was unstable, taking into account the relatively large length of the arm.
	
		\subsection{Ice Recovery and Mobility Agent}
		The pin joint design was chosen because the design has the best sturdiness, manufacturability, and because of its absence of a sliding motion. The bolt in the pin joint mechanism makes the retrieval mechanism sturdy in the direction going into and out of the page, perpendicular to the scooping motion. The rod connecting the drone to the scoop mechanism keeps a point on the scoop fixed to the drone at all times so the scoop is not flimsy. The connecting rod creates sturdiness in the direction parallel to the scooping motion because the rod is a fixed length. This allows control over how much the scoop can open or close based on how high the bottom of the connecting rod is on the bolt. Every part on the pin joint mechanism is thick with few potential points for failure. 

		The pin joint mechanism was also chosen due to its ability to convert rotational motion into translational motion. Instead of moving a mechanism up and down, which would not be feasible given the space inside the rocket, the team decided that a rotating motion to actuate the scoop would be ideal. This way there can just be a servo at the top instead of creating a new mechanism to slide up and down a railing. The team decided that having a nut on the bolt would be easiest to manufacture and simplest because it requires the least amount of moving parts, and the tolerances are easiest to fit into the bolt. The pin joint idea also allows for the easiest movement without friction because the moving part is the bolt which allows the mechanism to stay but move easily on command.

		The biggest risk to the current design is that the pin joint mechanism relies on an electrical servo which creates another method of failure if the avionics fail to perform as intended. Another problem is that the pin doesn’t entirely eliminate friction. The pin joint mechanism could theoretically get stuck if the servo force is not as strong as the frictional force on the pin. Another disadvantage is that there is little material on the other side of the pin from the scoop connecting the rod to the scoop. To compensate for this, strong material will be used like metal, and there will be testing performed to ensure that there is enough material with a safety factor.

		During the prototyping phase of the sliding pin joint mechanism, the team located a structural defect along the slit in the claw of IRMA. The team considered the possibility of the claw snapping along the slit when the mechanism is actuated. Furthermore, the team analyzed the feasibility of reducing friction between the joint and the claw and determined it to be impractical. After serious consideration, the team decided this method of ice sampling would not be manufacturable and structurally weak.

		The next iteration of IRMA, featuring the worm gear mechanism, had several issues. The cost of the worm gear mechanism was too high for further consideration. In addition, the weight of the mechanism did not satisfy the mass budget for the UAV. 

		After considering the restrictions of the other designs, the team has decided to move forward with the pin joint mechanism. Overall, the pin joint mechanism provides the most manufacturability and practicality of the considered designs. 


	\section{Avionics}\label{PL:Design:Avionics}
		\subsection{Flight Controller}
		The team examined the benefits and drawbacks of each flight controller: the Navio2, the ArduPilot, and the Pixhawk. Factors such as size and mass were taken into account but had minimal effect on the decision-making process. All three boards are compact, light and of a sturdy build. The differences could easily be accounted for by minor tweaks in the payload’s structure. Due to the irrelevance of these factors, the decision process was based predominantly by the computational capabilities of the controllers. 

		The ability to connect to the Raspberry Pi makes both the Pixhawk and the Navio2 more reliable than an individual ArduPilot due to the extensive experience that the drone community has with computer vision systems united with a Raspberry Pi. The ArduPilot was therefore eliminated as a viable option. 
		
		The Pixhawk and Navio2 were both viable controllers, each with their own benefits, such as higher reliability due to their communication with a Raspberry Pi and relatively easy computer vision integration because of the Raspberry Pi’s systems. One important difference between the Pixhawk and the Navio2 was how easy a computer vision system would be to manage. Although both systems have been used extensively for computer vision, the Pixhawk would require manual setup hardware-wise and with configuration files to establish a UART bridge. The Navio2, however, would require minimal setup due to its design including the Raspberry Pi as part of the flight controller. Both controllers would be able to provide the adequate amount of power that a computer vision guidance system would need, because of the Raspberry Pi’s high computing power and memory. 
		
		Ultimately, the Navio2’s reliability surpassed the Pixhawk’s because of the Navio2’s safer integration thanks to the extensive testing that third party organizations, such as Navio, have invested into their own products. The Navio2 was therefore selected as the flight controller most fitting for the mission parameters. 

		\subsection{Computer Vision}
		The team will be investigating an algorithmic and machine learning based approach of integrating computer vision in the vehicle’s guidance system. 

The team’s next steps towards deciding the computer vision implementation is to determine which computer vision method will control the drone. The team plans to start by programming both the algorithmic and deep learning approaches to computer vision, possibly through the use of the OpenCV or TensorFlow libraries on Python. Then, after both methods are programmed, the team aims to test them on the UAV.. This can be done by setting up an excavation area similar to the one to be used in the competition. Once the UAV is released, it should be able to navigate to the excavation area autonomously. Further extensive testing can be accomplished by putting “ice” samples into the excavation area and varying the ice depth at different points. Theoretically, once the drone reaches the excavation site, it should be able to recognize where the ice depth is greatest and then approach the site and retrieve the sample.

The final decision of which approach to implement for the competition will be based off of the results of these trials. The team will be testing for factors such as computational load, accuracy, speed, and reliability amongst others. The computer vision method with the best results will be further refined and used for the competition.

Using computer vision to guide the drone has numerous risks associated with it. With careful planning and testing, these risks can be accounted for to increase the probability of mission success.

