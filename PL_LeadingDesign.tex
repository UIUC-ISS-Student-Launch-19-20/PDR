\chapter{Leading Design of the Payload}

	\section{Deployment}\label{PL:Design:Deployment}
		The challenge posed to the team regarding the release of the payload is to deploy the UAV in the proper orientation for flight. The UAV requires a fast, consistent, and reliable release mechanism to deploy it from the rocket. During descent, the UAV will be deployed out of the top of the rocket as described below. 
		\subsection{Guide Rails}
			The UAV will be secured to the rocket by a system of guide rails. These rails will be connected to both the base of the fairing section and the retracted arms of the UAV. As the UAV is lowered out of the rocket, the arms will be released into the open position and ready for operation, pending approval from the RSO. The lowering of the UAV will be described by the winch system below.  

			\subsection{Lowering Mechanism}
				To lower the UAV out of the rocket body, the team has chosen to use a servo motor to power the winch. The team found this option suitable because it will allow the drone to be lowered at a constant speed within the safety margin. The team weighed the success of the mission over the complexities involved with integrating the servo powered winch system. In the case of electrical failure relating to the winch’s servo, the UAV will be lowered by gravity assisted by mechanical resistance from the dead servo. Leveraging the mechanical resistance, the servo-winch system will be within the safety margin and able to operate in the case of servo failure.

			\subsection{Lowering Mechanism Release System}
				To sever the winch line from the UAV, the team selected the method of burning a nylon string using an electric system. The other options were eliminated due to the complexity of the mechanical execution of the concepts; that is to say, the team feared that too many moving parts would lead to an overly complex release mechanism. The simplicity of severing a nylon string via heat appealed to the team because of its small risk of failure. The team plans on performing several rounds of tests to ensure that the electric system will be able to sever the nylon cord under any circumstances. 


	\section{Drone Structures}\label{PL:Design:Structures}
		\subsection{Arm Configuration}
			This is a filler paragraph

		\subsection{Landing Legs}
			This is a filler paragraph
	
		\subsection{Ice Recovery and Mobility Agent}
		The pin joint design was chosen because the design has the best sturdiness, manufacturability, and because of its absence of a sliding motion. The bolt in the pin joint mechanism makes the retrieval mechanism sturdy in the direction going into and out of the page, perpendicular to the scooping motion,  when looking at Figure INSERT FIGURE OF CAD OF PIN JOINT. The rod connecting the drone to the scoop mechanism keeps a point on the scoop fixed to the drone at all times so the scoop is not flimsy. The connecting rod creates sturdiness in the direction parallel to the scooping motion because the rod is a fixed length. This allows control over how much the scoop can open or close based on how high the bottom of the connecting rod is on the bolt. Every part on the pin joint mechanism is thick with few potential points for failure. 

		The pin joint mechanism was also chosen due to its ability to convert rotational motion into translational motion. Instead of moving a mechanism up and down, which would not be feasible given the space inside the rocket, the team decided that a rotating motion to actuate the scoop would be ideal. This way there can just be a servo at the top instead of creating a new mechanism to slide up and down a railing. The team decided that having a nut on the bolt would be easiest to manufacture and simplest because it requires the least amount of moving parts, and the tolerances are easiest to fit into the bolt. The pin joint idea also allows for the easiest movement without friction because the moving part is the bolt which allows the mechanism to stay but move easily on command.

		The biggest risk to the current design is that the pin joint mechanism relies on an electrical servo which creates another method of failure if the avionics fail to perform as intended. Another problem is that the pin doesn’t entirely eliminate friction. The pin joint mechanism could theoretically get stuck if the servo force is not as strong as the frictional force on the pin. Another disadvantage is that there is little material on the other side of the pin from the scoop connecting the rod to the scoop. To compensate for this, strong material will be used like metal, and there will be testing performed to ensure that there is enough material with a safety factor.



	\section{Avionics}\label{PL:Design:Avionics}
		\subsection{Flight Controller}
			This is a filler paragraph

		\subsection{Computer Vision}
			This is a filler paragraph
