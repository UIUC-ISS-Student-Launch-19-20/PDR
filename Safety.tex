\chapter{Safety Plan Overview}

The safety of all team members is of the absolute highest priority for the Illinois Space Society Student Launch team. Should a situation arise in which a project-critical choice needs to be made, safety is considered before the success of the project. The safety officer this year is Zana Essmyer, who is overseeing a small team to conduct a thorough analysis of any hazards the team may encounter this year throughout the design, construction, assembly, and launches of the rocket and payload. Zana and the safety team are also implementing plans and procedures to minimize the risk of associated hazards.

This year, using a combination of in-person briefings, online classes and thorough documentation, the team is actively encouraging participation in the adherence to safety procedures. Safety training is required for any member that wishes to participate in construction sessions or attend a launch. By keeping lists of safety-trained members and having experienced members actively involved at every build session, the team can ensure that everyone working in lab spaces understands safety protocol for both day-to-day work and potential emergency situations. Forthcoming checklists will also be developed to ensure total safety during the off-pad, on-pad and post flight procedures.

\section{Emergency Preparedness}

Though the Illinois Space Society strives to maintain a safe working environment during all phases of the competition, the team also recognizes that accidents remain a possibility even with the strictest safety precautions in place. With this in mind, emergency preparedness forms another pillar of the team's safety plan. First aid kits are easily accessible in all of the team's main workspaces, and the safety officer has familiarized herself with their contents. The kits themselves are up-to-date and include wound dressings, antibiotic ointments, painkillers, and antihistamines. For any injuries requiring more than basic first aid, medical facilities are available both on and off the University of Illinois campus.

\section{Incident Reporting}

In the rare event that an accident requiring first aid occurs, the primary goal is always to care for and assist the injured team member. That said, once the incident has passed, the safety team's next priority is to actively prevent accident reoccurrence. Any incident is to be reported immediately to the safety officer, and from there it will be her responsibility to speak to those involved and determine the exact cause of the accident. Review of an incident will be considered complete once the safety officer has surveyed the scenario to her satisfaction and offered recommendations to the team leadership on how to prevent similar incidents in the future.

If an incident happens to occur, a series of actions will enact. The non-injured team member will assess the situation for any immediate dangers before contacting the Safety Officer, Technical Manager, and, if needed, emergency personnel. The Safety Officer will document the incident. The safety team will then take action to prevent further incidents.

\section{Equipment Training}

In order to provide team members with the experience necessary to operate a wide array of equipment and tooling, the safety team provides tutorial sessions on all machinery and tooling that may be used during the course of construction that is provided in the Nuclear Engineering Laboratory. To that end, the safety team has duplicated or adapted manufacturer-provided operating procedures for these tools and uploaded them to the team's shared drive for easy access. A collection of all training documents can be found in APPENDIX D: ISS Common Materials and Equipment Training with information included for the following devices:

\begin{itemize}[noitemsep]
    \item Full Spectrum Laser Professional Series CO2 48"$\times$36"  Cutter
    \item  Ultimaker 2 Extended 3D Printer
    \item Milwaukee Sawzall Reciprocating Saw
    \item DeWalt 18V Wireless Power Drill
    \item Dremel 8200-1/28 12-Volt Max Cordless Rotary Tool
    \item G5000 RocketPoxy
    \item Grizzly Model G7297 12" Disc Sander
    \item Water-Cooled Diamond Table Saw
    \item Grizzly H2936 Vacuum Sanding Table
    \item GMC 16" Scroll Saw
    \item JET 15" Bench Drill Press
    \item Soldering Station
    \item Miscellaneous Hand Tools (Screwdrivers, Hammer, Clamps, etc.)
\end{itemize}

In order to operate this machinery, a team member must attend tutorial sessions or receive training separately from a member of the safety team or team management. In providing and requiring these sessions, the team not only reduces the risk of mishaps due to misuse of equipment, but also ensures redundancy in knowledge of construction techniques. The Safety Officer will host tutorial sessions for ESPL machinery at the beginning of the semester for all members before access will be granted.

\section{NAR/TRA Procedures}

The team will comply with the ``High Power Rocket Safety Code'' provided on the NAR website that has been effective since August 2012. The 13-step code and Minimum Distance Table on the website will be reviewed by the safety officer. All members on the team will be required to read the safety code online as it is a relatively short list of codes. The rules set forth by the NAR High Power Rocketry Code will always be respected and followed as they are set to ensure the safety of people and the environment. The safety officer, team manager, and sub-team managers will always make sure to comply with the safety code and ensure the rest of the team is properly complying. A copy of the NAR High Power Rocketry Code is included in this report as APPENDIX A: NAR High-Power Rocketry Safety Code. Additionally, sections ADDSEC and ADDSEC elaborate on the hazardous operations and mitigation of risks. Hazardous materials and proper protocol is detailed under section ADDSEC

\paragraph{NAR Mentor.}

Mark Joseph will be the NAR mentor for the ISS Student Launch team for this year's competition. In addition to his longtime involvement in the high-power rocketry community, Mark has worked with the ISS team for several years now in the NASA Space Grant, Intercollegiate Rocket Engineering, and Student Launch Competitions.

\chapter{Risk Assessment Overview}

To better prepare for issues that inevitably arise during any project of large scale and to prioritize the team's time, the safety team has conducted a thorough risk analysis based on incident severity. The safety team analyzed risks to the project, the environment, and above all, the health of team members during the construction process. The team used Risk Assessment Codes (RACs) to evaluate the various hazards to both personnel and the project. Table 6 introduces the risk matrix and the risk assessment codes that will be used to classify risks throughout the rest of the safety section. Risks are color-coded based on their severity, and  discusses the team's response to these various levels.  defines the levels of severity as it relates to personnel, project, and environmental health. Table 9 defines individual instance probability and probability of occurrence throughout the entire project timeline.

\begin{table}[H]
    \centering
    \caption{Level of risk and member requirements}
    \label{tab:level of risk and member requirements}
    \begin{tabularx}{0.8\linewidth}{X c c c c}
        \toprule
       \multirow{2}{*}{\textbf{Probability}} & \multicolumn{4}{c}{\textbf{Severity}} \\
       \cmidrule(l){2-5}
        & 1---Catastrophic & 2---Critical & 3---Marginal & 4---Negligible \\
       \midrule
       A---Frequent & \cellcolor{red!25} 1A & \cellcolor{red!25} 2A & \cellcolor{orange!25} 3A & \cellcolor{green!25} 4A \\
       B---Probable & \cellcolor{red!25} 1B & \cellcolor{red!25} 2B & \cellcolor{orange!25} 3B & \cellcolor{green!25} 4B \\
       C---Occasional & \cellcolor{red!25} 1C & \cellcolor{orange!25} 2C & \cellcolor{orange!25} 3C & 4C \\
       D---Remote & \cellcolor{orange!25} 1D & \cellcolor{orange!25} 2D & \cellcolor{green!25} 3D & 4D \\
       E---Improbable & \cellcolor{green!25} 1E & \cellcolor{green!25} 2E & \cellcolor{green!25} 3E & 4E \\
       \bottomrule
    \end{tabularx}
\end{table}