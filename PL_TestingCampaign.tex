\chapter{Testing Campaign}

\section{Structual Testing}\label{PL:Testing:Structural}
	\subsection{Deployment Tests}
		This is a filler paragraph

	\subsection{Ice Retrieval and Mobility Agent Tests}
		\subsubsection{Longevity Test}
		The Ice Retrieval and Mobility Agent will be tested by letting open and close repeatedly for an extended period of time. This ensures that the mechanism can loop through many times without any part getting unaligned. The test will also be performed and running over a long period of time so the team can observe the mechanical wear on the mechanism. The test of cycling the open and close motion also confirms that the mechanism performs the way as designed when turned on. 
		\subsubsection{Mechanism in Action Tests}
		The team will test the mechanism on an ice sample to ensure that the retrieval mechanism can acquire and hold an ice sample in the area where the scoops intersect. The first part is to ensure the ice retrieval and mobility agent can scoop the ice inside the mechanism instead of just pushing or sliding the ice to a different location. The amount of sample the claw can scoop in one trial will also be observed during this test. Additionally, the team will have the UAV transport the ice to a different location to test for sample leakages and design deficiencies.
		\subsubsection{Rigidity Test}
		One of the reasons the leading design was chosen was because of its structural integrity. To confirm this, the structural integrity of the mechanism will be tested by observing what happens when vibrations and forces in all directions are applied to the mechanism. The conditions the mechanism experiences in the testing phase will be far harsher than those it should experience in flight. A few tests may look like rigorously shaking the mechanism, pushing hard against the scoop simulating resistance from the ice, and pulling and pushing on the pins to confirm their strength. 
		\subsubsection{Optimization Tests}
		The final action will be to optimize the mechanism after seeing the mechanism work in practice. If any part of the mechanism does not work according to plan, the dimensions will be changed to maximize the amount of ice acquired in one scoop. The mechanism will also be optimized to minimize leakage and space the mechanism takes inside of the rocket.   

\section{Avionics Testings}\label{PL:Testing:Avionics}
	\subsection{Flight Controller Tests}
		This is a filler paragraph

	\subsection{Computer Vision Tests}
		This is a filler paragraph





