\chapter{Testing Campaign}

	This section outlines the various testing campaigns our team will preform on various aspects of the payload to ensure a maximum chance of mission success.

\section{Structual Testing}\label{PL:Testing:Structural}
	\subsection{Deployment Tests}
	Primary structural members such as the carbon fiber tubes will be tested to failure in order to ensure their compliance with structural integrity requirements. A safety margin of 300 percent will be enforced on all load bearing members. Furthermore, mechanisms designed to actuate during the mission will be cycled repeatedly to verify that joints and hinges are not susceptible to wear. The passive deployment system of the arm revolves around the spring system being used to pull the arms from their resting position inside the rocket during ascent. However, the spring system must be calibrated correctly to ensure that the spring force is large enough to pull down  the arms while at the same time being small enough to not impede the deployment of the drone. As a result, many different spring configurations with varying spring forces will be tested to find the correct calibration. The spring forces will be adjusted by either choosing springs with different spring constants or varying their extension distances. 

	These tests will include the deployment of the UAV at varying orientations and recording their ability to allow the UAV to separate from the fairing. In addition, the ability of the leg system to fold up with the arm will be tested to ensure that the folding mechanism does not catch on any part of the UAV during deployment. In order to do this, UAV deployment will be conducted at varying orientations and forces to see whether the leg mechanism is successfully deployed.
	
	One very critical aspect of the vehicle structure is the arm’s locking mechanism. Failure of this locking mechanism during flight will result in catastrophic failure, as the arms will fold up and render the vehicle uncontrollable. In order to mitigate the risk of failure of the locking system, the vehicle will be supported from under the motor mounts, and the frame center will be loaded to 300\% of the loads expected during the mission. Failure of the arms to stay in the locked configuration will indicate that the locking mechanism does not meet operational requirements, which requires that the design of the mechanism be modified. Once the locking mechanism is confirmed to meet the safety margin requirement, the arms will be repeatedly actuated to ensure the security of the locks do not degrade with time.
	
	The testing of the jettison of the nose cone will be similar to the ejection testing that the rocket will undergo. However, instead of using ejection charges, the compressed gas will be tested in order to ensure that the nose cone will be successfully deployed. The deployment will be tested in numerous orientations to demonstrate that the nose cone will be jettisoned regardless of its orientation during descent. 
	
	The ability of the lowering mechanism to properly lower and release the UAV from the rocket will be tested in several ways. To ensure that the UAV will be deployed regardless of the functionality of the servo, the winch will be tested numerous times to check that deployment can occur without the presence of the servo motor. To test the ability of the severing of the winch cable, the electric system that cuts the cord will be tested to see if it can cut a cord with double the thickness of the cord that will be used to attach the UAV to the rocket. In order to make sure that the UAV will be severed no matter the angle of deployment that the UAV is in, the winch system and the severing will be tested numerous times in various orientations. This will also indicate the possible ways in which tangling could occur, and will indicate how best to coil the winch cord in order to avoid tangling. 
	

	\subsection{Ice Retrieval and Mobility Agent Tests}
		\subsubsection{Structural Analysis}
		Structural analysis of IRMA will be based on perceived failure points of the mechanism and potential failure modes that could introduce new points of stress on the system.  From a  mathematics perspective, the team will use the virtual displacement method and Euler’s critical load to understand the force that can be withstood by the IRMA structures.  These methods will let the team understand how the constraints and elements of the system interact and cause failure.

		The team has perceived multiple locations that are more likely to result in a structural failure.  These points include all pin connections,the potential for excessive bending of the connections to the UAV’s main body, bending of the claws, and shear failure of the member that connects the claw to the nut and the claw to the drone body.  Parallel to the above points of analysis, the team will also analyse how IRMA could handle landing directly on an open claw.  The team will also use the methods mentioned above to validate the necessary rigidity of all IRMA elements.
		
		\subsubsection{Longevity Test}
		The Ice Retrieval and Mobility Agent will be tested by letting open and close repeatedly for an extended period of time. This ensures that the mechanism can loop through many times without any part getting unaligned. The test will also be performed and running over a long period of time so the team can observe the mechanical wear on the mechanism. The test of cycling the open and close motion also confirms that the mechanism performs the way as designed when turned on. 
		\subsubsection{Mechanism in Action Tests}
		The team will test the mechanism on an ice sample to ensure that the retrieval mechanism can acquire and hold an ice sample in the area where the scoops intersect. The first part is to ensure the ice retrieval and mobility agent can scoop the ice inside the mechanism instead of just pushing or sliding the ice to a different location. The amount of sample the claw can scoop in one trial will also be observed during this test. Additionally, the team will have the UAV transport the ice to a different location to test for sample leakages and design deficiencies.
		\subsubsection{Rigidity Test}
		One of the reasons the leading design was chosen was because of its structural integrity. To confirm this, the structural integrity of the mechanism will be tested by observing what happens when vibrations and forces in all directions are applied to the mechanism. The conditions the mechanism experiences in the testing phase will be far harsher than those it should experience in flight. A few tests may look like rigorously shaking the mechanism, pushing hard against the scoop simulating resistance from the ice, and pulling and pushing on the pins to confirm their strength. 
		\subsubsection{Optimization Tests}
		The final action will be to optimize the mechanism after seeing the mechanism work in practice. If any part of the mechanism does not work according to plan, the dimensions will be changed to maximize the amount of ice acquired in one scoop. The mechanism will also be optimized to minimize leakage and space the mechanism takes inside of the rocket.   

\section{Avionics Testings}\label{PL:Testing:Avionics}
	\subsection{Flight Controller Tests}
	The flight controller is the most important component of the vehicle’s guidance and control system. The team will be using a Nabio2 board that  interfaces with a Raspberry Pi which will be running an instance of the ArduPilot open source flight controller software. Although the ArduPilot framework is very widely used and is proven to be effective and reliable, multiple parameters must be precisely determined to ensure the drone is able to fly in a stable manner.

	Before any flight tests can be done, the flight controller must be calibrated and configured correctly. The ArduPilot software incorporates built-in calibration procedures and health checks for all sensors and periphery. If any anomalous readings are detected, or if calibration is not successful, the software will notify the team through telemetry and will not allow arming of the motors. Once all sensors are calibrated, the motors will be armed and spun without any propellers mounted. Manually tilting the vehicle while armed should result in the motors varying in speed in an attempt to bring the vehicle’s attitude back to level with the horizon. If all motors are confirmed to produce the appropriate reactions, propellers can be mounted and a maiden flight of the drone can be conducted.
	
	The drone’s maiden flight will be done under the control of a human operator through an RC link. The purpose of the maiden flight shall be to verify the stability and controllability of the vehicle under manual control. During the flight, the telemetry feed will be under observation for any anomalous readings from sensors and GPS. If no anomalies are observed and the vehicle is deemed safely controllable, another flight will be made with the purpose of testing the flight controller’s autonomous capabilities.
	
	During this second flight, the vehicle will take off under manual control of a human operator. The flight controller will then be put into loiter mode which will attempt to maintain the drone’s position in space. If the vehicle is observed to be maintaining its position effectively, the flight controller will then be put into mission mode. In this mode, the vehicle will attempt to follow a predetermined set of waypoints, land autonomously, take off autonomously and travel to a different landing site at least 10 feet away. Data received through telemetry will be consistently observed for anomalous readings during this test mission. If no anomalies are detected and the vehicle accomplishes the test mission effectively, the flight controller will be considered to be working as intended.
	

	\subsection{Computer Vision Tests}
	The team’s primary concern regarding the computer vision algorithm is the susceptibility of the algorithm to detect and track  a false target. As a failsafe, the UAV’s guidance algorithm will be configured to revert to GPS in the case of the computer vision algorithm failing to detect the retrieval site. However, a false detection could result in the vehicle’s descent being guided in the wrong direction, jeopardizing the chance of mission success.

	In order to mitigate the risk of a false detection and incorrect guidance, the computer vision algorithm will undergo extensive testing and verification both on the ground and during flight. The sample retrieval sites are specified to be located within a 10 x 10 foot yellow tarp, however environmental conditions like the sun’s intensity,cloud cover, and general discontinuities will affect the exact gradient of yellow perceived by the onboard camera. Furthermore, a non overhead perspective will result in the square yellow tarp being perceived as trapezoid or a parallelogram by the algorithm.

	To verify that the computer vision algorithm is not negatively affected by the aforementioned corner cases, the algorithm will be tested on static images emulating what the onboard camera would be capturing during flight. To further verify the algorithm’s robustness, the algorithm will run  during flight with no guidance input to the flight controller. This algorithm will instead record the images captured and the guidance output it is creating. This will allow the team to compare the output from the computer vision algorithm to the images taken by the onboard camera. If the algorithm’s output is congruent to the location of the sample site in the images, the team will progress onto the next phase of testing.

	The final and most elaborate testing phase will involve allocating the computer vision system’s output to the guidance system. This allows the vehicle to be guided by the algorithm during descent onto the sample retrieval site. The test will be conducted with a human operator ready to override control of the vehicle remotely at any moment through an RC link. If the computer vision algorithm is able to effectively guide the UAV’s descent onto the retrieval site, it will be considered to be working as intended.






