\chapter{Vehicle Criteria}

\section{Launch Vehicle Overview}
    
    \subsection{Mission Statement}
    
    \subsection{Mass Statement}
    
    \subsection{Computer Aided Design}
    

\section{Booster Subsystem}

    \subsection{Motor Selection}

    \subsection{Motor Casing}


    \subsection{Motor Mount Assembly}


    \subsection{Motor Retention}


    \subsection{Outer Frame Dimensions}


    \subsection{Fins}


    \subsection{Launch Rail Integration}
    
\section{Payload Bay Subsystem}

    \subsection{Nose Cone}
    
    \subsection{Fairing Selection and Justification}
    
    \subsection{Interstage}
    
    \subsection{Manufacturing}
    
    \subsection{Outer Airframe}
    
\section{Recovery Subsystem}
The recovery subsystem is a vital component of the launch vehicle, allowing the hardware of the rocket to land at a safe speed, while also ensuring the safety of bystanders. This subsystem also accounts for any drift distance so that the most efficient safety measures can be employed. In this case, parachutes were selected for use as the most efficient and safest means of descent and recovery.

The rocket will use two parachutes: a main parachute and a drogue parachute. The main parachute will be a Fruity Chutes Iris Ultra with a diameter of 72 inches. The drogue parachute will also be from the Fruity Chutes company and will have a diameter of 20 inches. This brand of parachutes was chosen due to the team’s familiarity with the company and its products as used in previous rocket-building experiences.


    \subsection{Deployment Scheme}
The parachutes will be deployed in two main separation events. The first will occur one second after apogee, which will separate the booster section from the switch band and release the drogue parachute. After this separation, the two sections will remain attached by the shock cord, which is in turn attached to the drogue parachute. This event occurs at a high enough altitude to ensure the rocket will slow enough to safely withstand the force of the release of the main parachute.

The second separation will occur at 800 feet of altitude. This separation will be between the switch band and the upper body tube. This will deploy the main parachute. This altitude is high enough to slow the rocket to a safe landing speed and minimize midair drift, and it is low enough to adhere to the competition limit of 1,000 feet.
 
The parachutes will be deployed in two main separation events via black powder ejection charges. The reason for use of black powder is so that we can time our separations for a safer descent. 

    \subsection{Deployment Mechanism}
The motor does not come with an ejection charge, and more than one separation is needed, hence we will be using black powder ejection charges. Using black powder allows for the separations to be conducted at specific times so that they can happen at desirable altitudes. The charges will be stored in small canisters that are ignited by e-matches that activate with a specific electrical signal received from the altimeters. The pressure from the ignited black powder explosion leads to a force that shears the pins holding the sections together, freeing them to move and separate. 

An advantage of using black powder is that we can control the amount of black powder used for each separation event and when it is ignited. Thus we can control the force of the charge so that the sections separate totally and safely, and we can have several charges, each occurring at different times at the team’s discretion. This flexibility allows management of the descent of the rocket with the staggered deployment of the drogue and main parachutes at safe altitudes and speeds. Based on the size and weight of the launch vehicle and each section, estimations of how much black powder we will need to use will be calculated. Through multiple trials, the amount of black powder that is used in the ejection charge will be refined. Preliminary calculations for black powder will be made using the relevant equations for the ideal gas law, as shown below.

    \subsection{Parachute Sizing}
    
    \subsection{Redundancy}
The redundancy of altitude measurements and deployment systems is critical to mission accomplishment and safety assurance. Consistent and accurate measurement of altitude will allow for expected performance of the ejection and deployment systems, and as a result will maximize safety of observers and hardware, and mission success.

Independent altimeters will control their respective ejection charges, and each altimeter will receive power from its own power supply for optimal safety and failure avoidance. The two ejections will each be triggered by independent primary and secondary charges; the secondary charge will be larger than the initial charge to ensure that separation is achieved. 



\section{Mission Performance}

    \subsection{Simulations Methods}
    
    \subsection{Flight Profile Analysis}
    
    \subsection{Stability}
    
    \subsection{Kinetic Energy}
    
    \subsection{Drift}
    
    \subsection{Target Altitude Statement}

\section{Subscale}

    \subsection{Overview}
    
    \subsection{Mass Statement}
    
    \subsection{Motor Selection}
    
    \subsection{Outer Airframe}
    
    \subsection{Fins}
    
    \subsection{Avionics}
    
    \subsection{Payload}
    
    \subsection{Recovery}
